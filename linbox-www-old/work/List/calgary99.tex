% LaTeX2e document
\documentclass[12pt]{article}

\usepackage{url}
%%%\usepackage{mathptm}

% NSF margin requirements
\setlength\oddsidemargin{2.5cm}
\addtolength\oddsidemargin{-1in}% makes left margin exactly 2.5cm
\setlength\evensidemargin{2.5cm}
\addtolength\evensidemargin{-1in}
\setlength\textwidth{8.5in}
\addtolength\textwidth{-5cm}% makes right margin exactly 2.5.cm
%
\setlength\textheight{11in}
\addtolength\textheight{-5cm}% text has 2.5cm top and bottom margins
% note: 1in + \voffset + \topmargin + \headheight + \headsep = 2.5cm
\voffset=-1.0in
\setlength\topmargin{1cm}
\setlength\headheight{1cm}
\setlength\headsep{0.5cm}
\setlength\footskip{0.6cm}

\newcommand{\probf}{\scshape}
\newcommand{\tr}{{\text{Tr}}}
\newcommand{\implies}{\implies}
\newcommand{\open}{$\underline{\smash{\hbox{open}}}$\xspace}
\newcommand{\conjecture}{CONJECTURE\xspace}

\newenvironment{problem}[1]
{\goodbreak\vspace{\medskipamount}\noindent
{\probf #1}\nobreak\newline
}%
{\par\vspace{\medskipamount}\hrule
}% end newenv problem

\newenvironment{statement}%
{{\bfseries Problem statement\ \ }\begingroup\itshape}
{\endgroup}% end newenv statement
\newcommand{\smallfields}{\par
   \vspace{\medskipamount}\noindent{\bfseries Small fields\ \ }}
\newcommand{\bigfields}{\par
   \vspace{\medskipamount}\noindent{\bfseries Big fields\ \ }}
\newcommand{\specialcase}[1]{\par
   \vspace{\medskipamount}\noindent{\bfseries Special case}~{\probf #1}\ \ }

\usepackage{newlfont}
\usepackage{xspace}
\usepackage{array}
\usepackage{amsfonts}
\usepackage{amsmath}

% BibTeX styles
%%%\usepackage{natbib}
%%%\bibpunct[,]{[}{]}{,}{a}{}{,}
%%%\newcommand{\citeyearp}[1]{[\citeyear{#1}]}

%%%bibstyle{plain}
\newcommand{\citep}{\cite}
\newcommand{\citeyearp}{\cite}

\usepackage{epsfig}


\begin{document}
%
\title{Linear Algebra With Black Box Matrices\\
       A List of Problems and Results}
\author{Wayne Eberly\quad Erich Kaltofen\quad Gilles Villard}
\date{July 27, 1999}
\maketitle
%
%

\section*{Linear system solving}

\begin{problem}{LinSolve0}
\begin{statement}
   Given $A$, compute $w \ne 0$ such that $Aw = 0$.
\end{statement}

\noindent $\implies$ {\probf LinSolve1} for non-singular $A$:
solve $[A \mid b] w = 0$.

\noindent $\implies$ singularity certificates and a Monte Carlo
   test for non-singularity: if any of the algorithms repeatedly fails,
   the matrix likely is non-singular.

\smallfields  Block Wiedemann \citep{Co94} together
   with tricks in \citep{Ka95:mathcomp}
   leading to $(1 + \varepsilon)n$ or $(2 + \varepsilon)n$ matrix-times-vector 
   products, the latter if the left blocking factor is kept 1.
   The complete analysis without using any matrix pre-conditioners
   is in \citep{Vil97,Vil97:TR, Vil98}.
\noindent  Note that in terms of number of matrix-times-vector products,
   block Wiedemann seems to beat the block Lanczos variants because Step~C3
   in \citep{Ka95:mathcomp} with a single vector
   seems cheaper than the bi-directional Lanczos projections with full blocks.
   However, (block) Lanczos incorporates the early termination strategy first
   observed by \cite{Lo98}: if the linear generator is short, the solution is
   found without completing the sequence to the worst case length.
   If this is the case, block Lanczos may be superior.

\bigfields Lanczos-like block Wiedemann; \cite{Eb99} shows that look-ahead
   is unlikely.

\specialcase{Sym}
   $A$ is symmetric.\footnote{%
The intent is to append the special case designation to the problem name,
such as {\probf LinSolve0Sym}.
}
   Unblocked Lanczos, because it only
   requires $n$ matrix-times-vector products in the worst case.
   For fields of characteristic zero, an issue is the growth of the
   size of the scalars, for which blocking may be helpful.

\specialcase{WithTr} It is \open how an additional black box for $A^\tr$
   can improve the above methods.

\end{problem}% LinSolve0


\begin{problem}{LinSolve1}
\begin{statement}
   Given $A$ and $b$,
   compute $x$ such that $Ax = b$.
\end{statement}

\noindent $Aw = 0$, where $w \in \text{right-nullspace}(A)$
uniformly selected, is reducible to {\probf LinSolve1}: $Ax = Ay$,
where $y$ is a random vector.  In the reverse direction,
consider $[A \mid b] w = 0$.
It is \open how to prove the reduction to {\probf LinSolve0}
(cf.\ the remark after the proof of theorem~8 in~\citep{Ka95:mathcomp}).

\noindent {\probf Rank} + {\probf PreCondrxr} solve this problem
\citep{KaSa91}.

\smallfields {\probf PreCondNil} together with block Wiedemann
as discussed under {\probf LinSolve0}.
 
\bigfields {\probf PreCondNil} together with Lanczos-like
block Wiedemann as discussed under {\probf LinSolve0}.
 
\specialcase{NonSing} $A$ is non-singular: no {\probf PreCondNil} is needed.

\specialcase{Incons} Certificates for inconsistency are known only
with an additional black box for $A^\tr$ \cite{GLS98,Ka98:linbox}.
Without a transpose box, the problem is \open.
 
\end{problem}% LinSolve1


\section*{Pre-conditioners}

The following problems concern pre-conditioning of the coefficient
matrix.  Here it is assumed that the pre-conditioner can be used, at least,
for deriving solutions of a system with a black box coefficient matrix.

\vspace{\bigskipamount}
\begin{problem}{PreCondNil}
\begin{statement}
   Given a singular matrix $A$,
   pre-condition to $\tilde A$
   such that $\tilde A$ has no nilpotent blocks in its Jordan canonical form.
\end{statement}

\noindent See {\probf LinSolve1}.
 
\smallfields $\tilde A = W_1 \cdot A \cdot W_2$ \citep{Vi99},
   where $W_i$ are the sparse matrices constructed by \cite{Wie86}.
 
\bigfields See also {\probf PreCondGen}.
 
\specialcase{Sym} For big fields,
$\tilde A = D \cdot A \cdot D$,
where $D$ is a random diagonal matrix \citep{EVS99}.
\end{problem}% PreCondNil

\begin{problem}{PreCondrxr}
\begin{statement}
   Given $A$ of rank $r$ (unknown),
   pre-condition to $\tilde A$ such that
   the $r \times r$ leading principal minor of $\tilde A$ is non-zero.
\end{statement}

\noindent See {\probf LinSolve1}.

\smallfields  $\tilde A = W_1 \cdot A \cdot W_2$
   where $W_i$ are the sparse matrices constructed by \cite{Wie86}.

\bigfields See also {\probf PreCondGen};
the failure probabilities are smaller.
\end{problem}% PreCondrxr

\begin{problem}{PreCondsxs}
\begin{statement}
   Given $A$ of rank $r$ (unknown),
   and given $s \le r$,
   pre-condition to $\tilde A$
   such that the $s \times s$ leading principal minor of
   $\tilde A$ is non-zero.
\end{statement}

\noindent See {\probf Rank}.

\smallfields  $\tilde A = W_1 \cdot A \cdot W_2$
   where $W_i$ are the sparse matrices constructed by \cite{Wie86}.
 
\bigfields See also {\probf PreCondGen};
the failure probabilities are smaller.
\end{problem}% PreCondsxs

\begin{problem}{PreCondGen}
\begin{statement}
   Given $A$,
   pre-condition to $\tilde A$
   such that $\tilde A$ has generic rank profile, i.e., all principal
   minors up to the rank of the matrix are non-zero.
\end{statement}

\noindent $\implies$ {\probf PreCondsxs, PreCondrxr}.
 
\smallfields \open
 
\bigfields $\tilde A = B_1 \cdot A \cdot B_2$, where $B_i$ encode symbolic
Bene\v{s} permutation networks \citep{Wie86}.

\vspace{\smallskipamount}\noindent
$\tilde A = T_{\text{upper}}\cdot A\cdot T_{\text{lower}}$, 
where $T_{\text{upper}}$ is a random unit upper triangular Toeplitz
matrix and $T_{\text{lower}}$ is a random unit lower triangular Toeplitz
matrix \citep{KaSa91}.

\specialcase{NonSing} The above pre-conditioners will work with
a single multiplier on either side.
\end{problem}% PreCondGen

\begin{problem}{PreCondSqufree}
\begin{statement}
   Given $A$ singular,
   pre-condition to $\tilde A$
   such that $\text{rank}(\tilde A) = \text{rank}(A)$,
   $\text{min-poly}(\tilde A) = f(x) \cdot x$ where $f$ is
   squarefree and $f(0) \ne 0$, and
   $\text{char-poly}(\tilde A) = f(x) \cdot x^k$.
\end{statement}

\noindent $\implies k = n - \text{rank}(A)$ \cite{KaSa91}; see {\probf Rank}.

\noindent $\implies$ {\probf PreCondNil, PreCondSqufree-x}.

\smallfields \open
 
\bigfields $\tilde A = \text{\probf PreCondGen}(A) \cdot D$, where
$D$ is a random diagonal matrix \citep{KaSa91}.
 
\specialcase{WithTr} $\tilde A = A^\tr \cdot D \cdot A$, where $D$ is a random diagonal matrix,
if an additional black box for $A^\tr$ is given
and the field is large \citep{EbKa97}.

\end{problem}% PreCondSqufree

\begin{problem}{PreCondSqufree-x}
\begin{statement}
   Given $A$,
   pre-condition to $\tilde A$
   such that $\text{min-poly}(\tilde A) = f(x) \cdot x^l$ where $f$ is
   squarefree and $f(0) \ne 0$, and
   $\text{char-poly}(\tilde A) = f(x) \cdot x^k$.
\end{statement}

\specialcase{BigChar} The characteristic of the coefficient field
is~$0$ or $> n$.
$\tilde A = D \cdot A$, where $D$ is a random diagonal matrix
\citep{Sau98}.\newline
Note that for large fields of small positive characteristic,
the pre-conditioner described in the previous sentence
might loose squarefreeness of $f$,
but not the other property (see {\probf PreCondCharPolyBigChar}).

\end{problem}% PreCondSqufree-x

\begin{problem}{PreCondCharPoly}
\begin{statement}
   Given $A$ non-singular,
   pre-condition to non-singular $\tilde A$
   such that $\text{char-poly}(\tilde A) = \text{min-poly}(\tilde A)$.
\end{statement}

\noindent See {\probf Det}. The $\det(A)$ must be easily derivable
from $\det(\tilde A)$.

\smallfields \open
 
\bigfields $\text{\probf PreCondGenNonSing}(A) \cdot D$, where $D$ is a random
diagonal matrix \citep{Wie86}.

\vspace{\smallskipamount}\noindent
$\tilde A = T_{\text{upper}}\cdot A\cdot T_{\text{lower}}$,
where $T_{\text{upper}}$ is a random unit upper triangular Toeplitz
matrix and $T_{\text{lower}}$ is a random, non-unit, lower triangular Toeplitz
matrix \citep{KaPa92}.

\specialcase{BigChar} $\tilde A = D \cdot A$ if the characteristic of the field
is~$0$ or $> n$ \citep{Sau98}.
\end{problem}% PreCondCharPoly

\newpage
\section*{Additional matrix problems}

\begin{problem}{MinPoly}
\begin{statement}
   Given $A$,
   compute its minimum polynomial.
\end{statement}
 
\smallfields Original Wiedemann \citeyearp{Wie86}.
 
\bigfields Original Wiedemann \citeyearp{Wie86}.
It is \open how to recover the minimum polynomial from
the blocked versions, or how to certify the result.
 
\end{problem}% MinPoly

\begin{problem}{Det}
\begin{statement}
   Given $A$,
   compute $\det(A)$.
\end{statement}
 
\smallfields \open, except for ${\mathbb F}_2$: use
the singularity tests mentioned in {\sc LinSolve0}.\newline
It is also \open how to certify $\det(A) \ne 0$.
 
\bigfields {\probf PreCondCharPoly} + {\probf MinPoly} \citeyearp{Wie86}.
 
\end{problem}% Det

\begin{problem}{Rank}
\begin{statement}
   Given $A$,
   compute the rank of $A$.
\end{statement}
 
\smallfields Use {\probf PreCondsxs} and the probabilistic singularity
tests mentioned under {\probf LinSolve0} to find the rank by binary
search \citep{Wie86}.\newline
It is \open how to avoid the $O(\log n)$ many calls to the sparse solver.
 
\bigfields Also, {\probf PreCondSqufree} + {\probf MinPoly}~\citep{KaSa91}.
\newline
For fields of characteristic~$0$, the rank can be certified~\cite{SSV99}.
 
\end{problem}% Rank

\begin{problem}{CharPoly}
\begin{statement}
   Given $A$,
   compute the characteristic polynomial of $A$.
\end{statement}
 
\smallfields \open
 
\bigfields \open
\end{problem}% CharPoly


\section*{References}
The papers co-authored by Kaltofen can be retrieved on the Internet through links from
my homepage with URL \url{www.math.ncsu.edu/~kaltofen}.
\renewcommand{\refname}{\vspace{-1cm}}% suppresses \section*{References}
%%%\bibliographystyle{plainnat}
\bibliographystyle{plain}
\bibliography{strings,kaltofen,new,linbox,crossrefs,xref}

\end{document}
