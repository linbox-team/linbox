\section{Improving \linbox matrix multiplication}\label{sec:matmul}
%
Efficient matrix multiplication is key to \linbox library.
%
\subsection{Plugin}
%
We propose the following design pattern (the closest design pattner to our
knowledge is the \emph{strategy} one, see also \cite[Fig 2.]{Cung:2006:TC}.
The main advantage of this design pattern is that the modules always call back
the controller so that the best choice is always chosen.  Besides modules can
be easily added as \emph{plug-ins}.  An analogy can be drawn with dynamic
systems---once the controller sends a correction to the system, it receives
back a new measure that allows for a new correction.
%
\par
%
\begin{figure}[htbp]
\tikzstyle{block} = [draw, fill=blue!20, rectangle,
    minimum height=2em, minimum width=6em, rounded corners]
\tikzstyle{input} = [coordinate]
\tikzstyle{output} = [coordinate]
\tikzstyle{fleche} = [draw,->,shorten <=2pt, shorten >=2pt,thick]
% \tikzstyle{pinstyle} = [pin edge={to-,thin,black}]
        \centering
\begin{tikzpicture}[auto, node distance=2.5cm,>=latex']
        \node [input, name=input] (input) {};
        \node [block, right of=input] (controller) {Controler};
        \node [output, right of=controller] (output) {};

        \node [block, below of=controller] (modules) {Modules};

        \draw [fleche] (input) to [near start] node {input} (controller);
        \draw [fleche] (controller) to [near end] node {output} (output);
        \draw [fleche] (controller.300) to [bend left,near end] node [right] {call} (modules.60);
        \draw [fleche] (modules.120) to [bend left,near end] node [left] {call back} (controller.240);
\end{tikzpicture}
\caption{Controller--Modules design pattern}
\label{fig:diag:patt}
\end{figure}
%
For instance, we can write the standard cascade algorithms in that model:
%
\par
%
\begin{figure}[htbp]
        \hfil
        \begin{minipage}{0.45\textwidth}
                \begin{algorithm}[H]
                        \caption{\texttt{AlgoThresh}: controler}
                        \label{alg:controle}
                        \KwIn{$A$ and $B$, denses, with resp. dimensions $n\times
                        k$ and $k\times n$.}
                        \KwOut{$C = A\times B$}
                        \eIf{$\mathrm{min}(m,k,n)<t$}{ {\tt
                        BaseCase}(C,A,B) \tcc{\small{\color{gray}fast BLAS}} }
                        { {\tt RecursiveCase}(C,A,B,t)  }
                        \KwRet C \;
                \end{algorithm}
        \end{minipage}
        \hfil
        \begin{minipage}{0.45\textwidth}
                \begin{algorithm}[H]
                        \DontPrintSemicolon
                        \caption{\texttt{RecursiveCase}: recursive module}
                        \label{alg:action}
                        \KwIn{$A$, $B$, $C$ and $t$ as in controller.}
                        \KwOut{$C = A\times B$}
                        Cuts $A$,$B$,$C$ in $S_i, T_i\cdots$\;
                        \ldots \;
                        $P_i = \mathtt{AlgoThresh}(S_i,T_i,t)$ \;
                        \ldots \;
                        \KwRet C \;
                \end{algorithm}
        \end{minipage}
        \hfil
        \caption{Conception of a recursive controlled algorithm}
        \label{fig:seuil}
\end{figure}
%
This method allows for the reuse of modules and ensures efficiency.
It is then possible to adapt to the architecture, the available modules,
the resources. The only limitation is that the choice of the module
should be done fast.
%
%
\danger timing old fgemm/plugin fgemm with no noticeable change ?
%
\subsection{New algorithms/infrastructure}
%
We introduce now several new algorithms that improve on matrix multiplication
in various ways: reducing memory consumption, introducing new efficient
algorithms, using graphics capabilities, generalizing the BLAS to integer
routines.
%
\subsubsection{New algorithms: low memory}
%
\texttt{ffgem} in \fflas uses the classic schedules for the multiplication and
the product with accumulation (\cf \cite{Boyer:2009:sched}), but we also
implement the lower memory routines therein.
%
\par
%
The difficutlty consists in using the part of the memory contained in a
submatrix of the original matrix. It is two-fold. -- First we use some part
of a memory that has already been allocated to the input matrices, therefore
we cannot free and reallocate part of it. -- Second, several of these algorithms
are meant for square matrices and rectangular sub-matrices will just not be enough.
For instance,
%
\par
%
\danger table comparing speeds
%
\subsubsection{New algorithms: Bini}
%
 We recall
in \cref{tab:bini:322}\footnote{This table corrects some typo in the original
$W$ of the reference.} Bini's algorithm.
%
\begin{table}[htpb]
     \centering
   \begin{tabular}[H]{R!{$=$}LR!{$=$}LR!{$=$}L}
      X^{(1)} &
      \begin{pmatrix}
         1 & 0 \\
         0 & 1 \\
         0 & 0 \\
      \end{pmatrix} &
      Y^{(1)} &
      \begin{pmatrix}
         \epsilon & 0 \\
         0 & 1
      \end{pmatrix} &
      Z^{(1)} &
      \begin{pmatrix}
	      \epsilon^{-1} & 0 \\
         0 & 1 \\
         0 & 0
      \end{pmatrix} \\

      X^{(2)} &
      \begin{pmatrix}
         0 & 0 \\
         0 & 1 \\
         0 & 0
      \end{pmatrix} &
      Y^{(2)} &
      \begin{pmatrix}
         0 & 0 \\
         -1 & -1
      \end{pmatrix} &
      Z^{(2)} &
      \begin{pmatrix}
         \epsilon^{-1} & 0 \\
         0 & 0 \\
         0 & 0
      \end{pmatrix} \\

      X^{(3)} &
      \begin{pmatrix}
         1 & 0 \\
         0 & 0 \\
         0 & 0
      \end{pmatrix} &
      Y^{(3)} &
      \begin{pmatrix}
         0 & 0 \\
         0 & 1
      \end{pmatrix} &
      Z^{(3)} &
      \begin{pmatrix}
         -\epsilon^{-1} & -\epsilon^{-1} \\
         0 & 0 \\
         0 & 0
      \end{pmatrix} \\

      X^{(4)} &
      \begin{pmatrix}
         0 & \epsilon \\
         0 & 1 \\
         0 & 0 \\
      \end{pmatrix} &
      Y^{(4)} &
      \begin{pmatrix}
         -\epsilon & 0 \\
         1 & 0
      \end{pmatrix} &
      Z^{(4)} &
      \begin{pmatrix}
          \epsilon^{-1} & 0\\
         1 & 0 \\
         0 & 0
      \end{pmatrix} \\

      X^{(5)} &
      \begin{pmatrix}
         1 & \epsilon \\
         0 & 0 \\
         0 & 0
      \end{pmatrix} &
      Y^{(5)} &
      \begin{pmatrix}
         0 & \epsilon \\
         0 & 1
      \end{pmatrix} &
      Z^{(5)} &
      \begin{pmatrix}
         0 & \epsilon^{-1} \\
         0 & -1 \\
         0 & 0
      \end{pmatrix} \\

      X^{(6)} &
      \begin{pmatrix}
         0 & 0 \\
         1 & 0 \\
         0 & 1
      \end{pmatrix} &
      Y^{(6)} &
      \begin{pmatrix}
         1 & 0 \\
         0 & \epsilon
      \end{pmatrix} &
      Z^{(6)} &
      \begin{pmatrix}
         0 & 0 \\
         1 & 0 \\
         0 & \epsilon^{-1}
      \end{pmatrix} \\

      X^{(7)} &
      \begin{pmatrix}
         0 & 0 \\
         1 & 0 \\
         0 & 0
      \end{pmatrix} &
      Y^{(7)} &
      \begin{pmatrix}
         -1 & -1 \\
         0 & 0
      \end{pmatrix} &
      Z^{(7)} &
      \begin{pmatrix}
         0 & 0 \\
         0 & 0 \\
         0 & \epsilon^{-1}
      \end{pmatrix} \\

      X^{(8)} &
      \begin{pmatrix}
         0 & 0 \\
         0 & 0 \\
         0 & 1
      \end{pmatrix} &
      Y^{(8)} &
      \begin{pmatrix}
         1 & 0 \\
         0 & 0
      \end{pmatrix} &
      Z^{(8)} &
      \begin{pmatrix}
         0 & 0 \\
         0 & 0 \\
         -\epsilon^{-1}  & -\epsilon^{-1}
      \end{pmatrix} \\

      X^{(9)} &
      \begin{pmatrix}
         0 & 0 \\
         1 & 0 \\
         \epsilon & 0
      \end{pmatrix} &
      Y^{(9)} &
      \begin{pmatrix}
         0 & 1 \\
         0 & -\epsilon
      \end{pmatrix} &
      Z^{(9)} &
      \begin{pmatrix}
         0 & 0 \\
         0 & 1 \\
         0 & \epsilon^{-1}
      \end{pmatrix} \\

      X^{(10} &
      \begin{pmatrix}
         0 & 0 \\
         0 & 0 \\
         \epsilon & 1
      \end{pmatrix} &
      Y^{(10)} &
      \begin{pmatrix}
         1 & 0 \\
         \epsilon & 0
      \end{pmatrix} &
      Z^{(10)} &
      \begin{pmatrix}
         0 & 0 \\
         -1 & 0 \\
         \epsilon^{-1}  & 0
      \end{pmatrix} \\
   \end{tabular}
 \caption{Approximate $(3, 2, 2)$ Bini Formula.}
   \label{tab:bini:322}

\end{table}


%
The formula is recovered as this:
\begin{equation} \label{eq:base}
   %
   C = \sum_{r=1}^\mu \frob{A}{X^{(r)}} \frob{B}{Y^{(r)}} Z^{(r)}
   %
\end{equation}
where $\frob{\cdot}{\cdot}$ is the usual Frobenius product.
%
\par
%
It is well known (\cite{Bini:80:apa}) that approximate formulas for matrix
multiplication  yield exact matrix matrix multiplication formula.
This gives an approximate multiplication formula of type $(3, 2, 2)$ in $10$
multiplications (hence $\omega \approx 2.78$, in particular lower than
Strassen--Winograd).
%
\par
%
However, we can use this algorithm in several settings that seem to be new for
getting an exact algorithm.  We can for instance set $\epsilon = 2^{-26}$ (so
that $\epsilon^2 = 0$ for a \dbl) and then round off; or use $\epsilon = p$ for
computation in $\modring{p}$
%
\paragraph{Size of  $p$}
%
XXX
%
\paragraph{Implementation and timings}
%
XXX use something else than 101 (\flt...)
\begin{table}[htpb]
        \caption{XXX Algorithme approché de Bini \emph{\mbox{vs.}}
        Winograd (\texttt{fgemm}) sur $\modring{101}$ avec seuil de \numprint{1000}.}
        \label{tab:bini:fgemm}
        \centering
        \begin{tabular}{c cc}
                \toprule
                % \multirow{2}{*}{dimensions}
                & \multicolumn{2}{c}{temps (s)} \\
                \cmidrule(lr){2-3}
                 dimensions
                  &  \texttt{fgemm-bini} &  \texttt{fgemm-wino} \\
                \midrule
                $(\numprint{1080}, \numprint{1080}, \numprint{1080})$ & \numprint{0.33}  & \bf \numprint{0.32} \\
                $(\numprint{1800}, \numprint{1800}, \numprint{1800})$ &\numprint{1.37} & \bf \numprint{1.35} \\
                $(\numprint{2700}, \numprint{2700}, \numprint{2700})$ &\bf \numprint{4.38}& \numprint{4.56} \\
                $(\numprint{2700}, \numprint{1800}, \numprint{1800})$ &\bf \numprint{3.18}& \numprint{3.24} \\
                $(\numprint{4500}, \numprint{1800}, \numprint{1800})$ &\bf \numprint{1.95} & \numprint{1.98} \\
                $(\numprint{4050}, \numprint{2700}, \numprint{2700})$ & \bf \numprint{6.35} & \numprint{6.75} \\
                \bottomrule
        \end{tabular}
\end{table}
% \subsubsection{New algorithms: tr-tr}
%
% [schedule/implementation]
%
\subsubsection{integer blas}
%
\danger pascal
%
\subsubsection{OpenCL}
%
\danger dave
\subsubsection{Using conversions}
- using flint for integer matmul is faster, even with conversion. Need better CRA implementation. \\
- implementation of Toom-Cook for GF(q)\\
- when does spmv choose to optimise ?
