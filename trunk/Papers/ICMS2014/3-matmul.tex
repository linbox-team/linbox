\section{Improving \linbox matrix multiplication}\label{sec:matmul}
%
% Efficient matrix multiplication is key to \linbox library.
%
\subsection{Plug-in structure}
%
We propose the following \emph{design pattern} (the closest design pattern to our
knowledge is the \emph{strategy} one, see also \cite[Fig 2.]{Cung:2006:TC}).
The main advantage of this design pattern is that the modules always call back
the controller so that the best choice is always chosen.  Besides modules can
be easily added as \emph{plug-ins}.
%
\begin{wrapfigure}{r}{0.5\textwidth}
% \begin{figure}[htbp]
	\vspace{-2em}
	\begin{center}
	\small
\tikzstyle{block} = [draw, fill=blue!20, rectangle,
    minimum height=1.7em, minimum width=5.5em, rounded corners]
\tikzstyle{input} = [coordinate]
\tikzstyle{output} = [coordinate]
\tikzstyle{fleche} = [draw,->,shorten <=3pt, shorten >=3pt,thick]
% \tikzstyle{pinstyle} = [pin edge={to-,thin,black}]
	% \centering
\begin{tikzpicture}[auto, node distance=2.0cm,>=latex']
        \node [input, name=input] (input) {};
        \node [block, right of=input] (controller) {Controlers};
        \node [output, right of=controller] (output) {};

        \node [block, below of=controller] (modules) {Modules};

        \draw [fleche] (input) to [near start] node {input} (controller);
        \draw [fleche] (controller) to [near end] node {output} (output);
        \draw [fleche] (controller.300) to [bend left,near end] node [right] {call} (modules.60);
        \draw [fleche] (modules.120) to [bend left,near end] node [left] {call back} (controller.240);
\end{tikzpicture}
\end{center}
	\vspace{-1em}
\caption{Controller--Module design pattern}
\label{fig:diag:patt}
	\vspace{-1em}
% \end{figure}
\end{wrapfigure}

%
An analogy can be drawn with dynamic
systems---once the controller sends a correction to the system, it receives
back a new measure that allows for a new correction.
%
%
%
\par
%
For instance, we can write (\Cref{fig:seuil}) the standard cascade algorithms
(see \cite{Dumas:2008:Flas}) in that model. Cascade algorithms are used to combine
several algorithms that are switched using thresholds, ensuring better
efficiency than any of the algorithms within used separately.
%
% \par
%
%
This method allows for the reuse of modules and ensures efficiency.
It is then possible to adapt to the architecture, the available modules,
the resources. The only limitation is that the choice of the module
should be done fast.
%
\begin{figure}[tbp]
	\small
        \hfil
	\begin{minipage}[t]{0.47\textwidth}
                \begin{algorithm}[H]
                        \caption{\texttt{Algo}: controler}
                        \label{alg:controle}
                        \KwIn{$A$ and $B$, denses, with resp. dimensions $n\times
                        k$ and $k\times n$.}
			\KwIn{$H$ Helper}
                        \KwOut{$C = A\times B$}
			\eIf{$\mathrm{min}(m,k,n)<H.{\tt threshold}()$}{
				% {\tt BaseCase} K() ; \\
				{\tt Algo}(C,A,B,{\tt BaseCase}()) ; %\tcc{\small{\color{gray}fast BLAS}}
			}
			{
				% {\tt RecursiveCase} H() ; \\
				{\tt Algo}(C,A,B,{\tt RecursiveCase}())
			}
                        \KwRet C \;
                \end{algorithm}
        \end{minipage}
        \hfil
	\begin{minipage}[t]{0.47\textwidth}
                \begin{algorithm}[H]
                        \DontPrintSemicolon
                        \caption{\texttt{Algo}: recursive module}
                        \label{alg:action}
                        \KwIn{$A$, $B$, $C$ as in controller.}
			\KwIn{$H$, {\tt RecursiveCase} Helper}
                        \KwOut{$C = A\times B$}
                        Cuts $A$,$B$,$C$ in $S_i, T_i\cdots$\;
                        \ldots \;
                        $P_i = \mathtt{Algo}(S_i,T_i,H)$ \;
                        \ldots \;
                        \KwRet C \;
                \end{algorithm}
        \end{minipage}
        \hfil
        \caption{Conception of a recursive controlled algorithm}
        \label{fig:seuil}
	\vspace{-2em}
\end{figure}

% \par
%
On top of this design, we have Methods/Helpers that allow (preferred) selection
of algorithms and cut short in the strategy selection.
%
% \par
%
% \danger timing old fgemm/plugin fgemm with no noticeable change ?
%
\par
%
This infrastructure also invites to modularise the code. For instance,
\fflasffpack has seen major  modularization (addition, scaling,
reduction,\ldots) Not only it allows to write code with hardly more line than
it takes for pseudo-code listings in \cite{Boyer:2009:sched} (compared to
$\approx 2.5\times$ on some routines before) but also it automatically brings
performance, because we can the separately improve on these modules and can
switch easier between them. Also, this reduces the lines of code, hence the
probability for bugs, eases their tracing/tracking, and allows for more
unit tests. Modularising the code comes at almost no cost because we may add
$O(1)$  operations that: Do not cost much compared to $O(n^2)$ or more
complexity of the modules; Allow early decisions and terminations (\eg by
testing against $0$, $\pm 1$, checking leading dimensions or increments);
Allow better code (AVX, SSE, copy--cache friendly operation--copy back,
representation switching,\ldots)
%
\subsection{New algorithms/infrastructure}
%
We show several new algorithms that improve on matrix multiplication
in various ways: reducing memory consumption, introducing new efficient
algorithms, using graphics capabilities, generalizing the BLAS to integer
routines.
%
\def\monitem#1{\par \textit{#1}\ }
\monitem{Reduced memory.}
%
The routine \fgemm in \fflas uses by default the classic schedules for the multiplication
and the product with accumulation (\cf \cite{Boyer:2009:sched}), but we also
implement the low memory routines therein. The new algorithms are competitive
and can reach sizes that were limiting.
%
% \par
%
One difficulty consists in using the memory space contained in a sub-matrix of
the original matrix, that one cannot free or re-allocate.
%
\monitem{Using Bini's approximate formula.}
%
In \cite{BD:2014:Bini}, we use Bini's approximate matrix multiplication formula
to derive a new algorithms that is more efficient that the Strassen--Winograd
implementation in \fgemm by $\approx 5-10\%$ on sizes \num{1500}--\num{3000}.
This is a cascade of Bini's algorithm and Strassen--Winograd algorithm and/or
the naïve algorithm (using BLAS). The idea is to analyse precisely the error
term in the approximate formula and make it vanish.
%
\monitem{Integer BLAS.}
%
In order to provide fast matrix multiplication with multiprecision integers, we
rely on multi-modular approach through the chinese remainder theorem. Our
approach is to reduce as much as possible to \fgemm. Despite, the existence of
fast multimodular reduction (resp.\ reconstruction) algorithm
\cite{VonzurGathen:1999:MCA}, the naive quadratic approach can be reduced to
\fgemm which makes it more efficient into practice.  Note that providing
optimized fast multimodular reduction remains challenging. This code is
directly integrated  into \fflas.

%
\monitem{Polynomial Matrix Multiplication over small prime fields.}
%
The situation is similar to integer matrices since one can use
evaluation/interpolation techniques through DFT transforms. However, the
optimized Fast Fourier Transform of \cite{Harvey:2014}  makes fast evaluation
(resp.\ interpolation) competitive into practice. We thus rely on this scheme
together with \fgemm for pointwise matrix multiplications. One can find some
benchmark of our code in \cite{GioLeb14}.
%
\monitem{Sparse Matrix--Vector Multiplication}
%
Sparse matrices are usually problematic because the notion of \emph{sparsity}
is too general \vs the specificity of real world sparse matrices: the
algorithms have to adapt to the shape of the sparse matrices.
%
There is a huge literature from numerical linear algebra  on \spmv (Sparse
Matrix Vector multiplication) and on sparse matrix formats, some of which are
becoming standard (COO, CSR, BCSR, SKY,\ldots).  In \cite{Boyer:2010:spmv} we
developed some techniques to improve the \spmv operation in \linbox. Ideas
include the separation of the $\pm 1$ for removing multiplications, splitting
in a sum (HYB for hybrid format) of sparse matrix  whose formats are
independant and using specific routines. For instance, on $\modring{p}$ with
word size $p$, one can split the matrix ensuring no reduction is needed  in the
dot product and call Sparse BLAS (from Intel \textsf{MKL} or Nvidia
\textsf{cuBLAS} for instance) on each matrix. One trade-off is as usual between
available memory, time spent on optimizing \vs time spent on \apply, and all
the more so because we allow the concurrent storarge of the transpose in an
optimised fashion, usually yielding huge speed-ups. This can be decided by
\emph{ad hoc.\ } optimizers.
% Helper structure to store a matrix as a sum of structures (HYB), possibly the
%
\monitem{Parallelisation}
Work on parallelizations using \textsf{OpenCL}, \textsf{OpenMP} or
\textsf{XKaapi} for dense or sparse matrix multiplication include
\cite{Boyer:2010:spmv,WST12,DGPZ14}.

\paragraph{}
These state-of-the-art algorithms, often inter-dependant, need to be selected
by the \mul solution, possibly using a Method/Helper parameter. The
Controller/Module model works particularly well here: cascading, switch
methods, algorithms selection is made easy. Besides, the choosing of a
particular algorithms or base case is never settled in advance: the library has
to be tuned wrt. available libraries, local performance of these libraries.

\begin{comment}
\monitem{Using conversions}
\begin{itemize}
	\item
 double->float
	\item
 using flint for integer matmul is faster, even with conversion. Need better CRA implementation (but with the plugins, we can do without our faulty code and just use flint).
	\item
 implementation of Toom-Cook for GF(q)
	\item
 when does spmv choose to optimise ?
	\item
 transition to benchmarking
\end{itemize}
%
\danger dave some words on the Domain architecture that you support and why it is best than functions (for instance for mul or apply) ?
\end{comment}
