%MARCH 1, 1998: 
%^^^^^^^^^^^^^
%Deadline for abstracts of contributed papers, minisymposia talks, and
%invited talks. Submit electronically to
%
%     ilas98@math.wisc.edu .
%
%(Also use this address for inquiries.)
%
%The abstract template for your editing is appended at the end of this
%message. 
%
%APRIL 1, 1998:
%^^^^^^^^^^^^^
%Early registration (including banquet and excursion). After April 1, the
%registration fee increases by $10. Send to
%
%    ILAS98-Madison
%    Dept. of Math./Van Vleck Hall
%    480 Lincoln Drive
%    UW-Madison
%    Madison, WI 53706 (USA)
%
%
%APRIL 27, 1998:
%^^^^^^^^^^^^^^
%Deadline for Housing Accommodation at Chadbourne Hall (conference
%headquarters). Send housing registration form to:
%
%   Conference Groups Office
%   University Housing
%   625 Babcock Drive
%   Madison, WI 53706-1213 (USA).
%
%Accommodations are also set aside at Howard Johnson's Hotel:
%
%   608-251-5511.
%
%Hope to see you all in June for the biggest and best ILAS conference ever!
%
%%%%%%%%%%% cut here %%%%%%%%%%%%%%%%%%%%%%%%
%%%%%%%%% abstract template %%%%%%%%%%%%%%%%%

\documentstyle{article}
\setlength{\textheight}{9.0in}
\setlength{\textwidth}{6.2in}
\addtolength{\topmargin}{-1.0in}
\addtolength{\oddsidemargin}{-0.5in}

\begin{document}
%Edit title
\title{Solving parametric linear systems}
%Edit author and address
\author{
Mark W. Giesbrecht\\
Depatment of Computer Science\\
University of Manitoba\\
Winnipeg,  Manitoba, R3T 2N2, Canada\\
%Edit out coauthor if no coauthor; insert another \and if a third
%coauthor
\and
B. David Saunders\\
Department of Computer and Information Science\\
University of Delaware\\
Newark, DE 19716, USA\\
}

\date{}
\maketitle

\noindent
%Edit speaker
{\it Speaker:} David Saunders:
%Edit email address
saunders@udel.edu

\medskip
\noindent
%Edit key words
{\it Key identifying words:} parametric linear system, computer
algebra, polynomial matrix.

\medskip
%Edit text of abstract

When solving the linear system $Ax=b$, where the entries of the matrix
$A$ and vector $b$ are polynomial expressions in several parameters,
computer algebra systems such as Maple and Mathematica will give the
``generic'' solution.  Often more is desired.  For example, at values
of the parameters which are not zeroes of the determinant of $A$, the
solution is easily given. But it is also desirable to know the rank
and nature of the solution for points in parameter space at which the
matrix is singular.

The problem of succinctly describing all solutions is computationally
intractable (it is P-space hard).  Even the size of the output can be
exponential in the size of the inputs.  Previously described solution
methods have required exponential time in the size of the matrix.  In
this paper we present two algorithms and provide an analysis to show
the computation time is bounded by a function which is polynomial in
the matrix size and in the degrees of the entries as polynomials in
the parameters.  It is exponential only in the number of parameters.
Thus, for any fixed number of parameters, the algorithms run in time
which is a polynomial function of the size of the input matrix and
right hand side vector (or matrix).
%This size includes the space needed for the entry expressions.

\end{document}
